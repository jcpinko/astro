\documentclass[12pt,epsf]{article}
%%%%%%%%%%%%%%%%%%%%%%%%%%%%%%%%%%%%%%%%%%%%%%%%%%%%%%%%%%%%%%%%%%%%%%%%%%%%%%%%%%%%%%%%%%%%%%%%%%%%%%%%%%%%%%%%%%%%%%%%%%%%

% \input astro.tex

% Page enlargement commands
\setlength{\topmargin}{-0.3in}
\setlength{\oddsidemargin}{0.0in}
\setlength{\textwidth}{7in}
\setlength{\textheight}{9in}

% \setlength{\baselineskip}{10pt}
% \setlength{\parskip}{-0.2in}
\renewcommand{\baselinestretch}{0.8}

% \usepackage[dvips]{color}
% Enables these features ...
% \textcolor{colorname}{text} %writes text in a color which can be specified 
% by name (black, white, red, green, blue or a color name you've defined), 
% RGB components, or grayscale. 
% \colorbox{colorname}{text} %writes text in a box with a colored background, can generate blinking
% text. 
% \fcolorbox{framecolor}{boxcolor}{text} %writes text in a colored frame. 
% \pagecolor{colorname} %sets the color of the page's background. 
% \definecolor{colorname}{color specification} %lets you define new color names.
% \definecolor{gold}{rgb}{0.85,0.66,0} 


\begin{document}

% Original Lecture and Date:
% Chapter 3 outline.
%

\noindent 

\begin{centering}
{\Huge Physics 231.  Week 2. Motion in One Dimension. }
\end{centering}
\bigskip 

\Large
Outline: Motion in 1D

\begin{enumerate}
\item Specifying the position of our "particle" (e.g., the center-of-mass
of a person, car, etc.)
\begin{enumerate}
	\item Give distance from a reference position
        \item [Ex.] A wall in a room containing two people could define x=0.  "Mo is located at x=5'." then means that Mo is 5' to the right of the wall.
	\item 1D problems are graphically shown on a number line 
	\begin{enumerate}
          \item the reference position is at x=0
	  \item it can be horizontal or vertical as befits the problem
	  \item there are 2 possible directions of motion, + (positive x) or -
	\end{enumerate}
	\item 2D and 3D problems are graphically shown on cartesian coordinate systems
	\item Distance vs displacement
	\begin{enumerate}
	  \item distance = path length = $d$
	  \item displacement = difference in final and initial positions = $\Delta \vec{x}$
	  \item $\Delta \vec{x} = \vec{x}_{fin} - \vec{x}_{in}$, where $\vec{x}_{in}$
and $\vec{x}_{fin}$ are the initial and final position vectors
	  \item distance is a scalar
	  \item displacement is a vector
	  \item in 1D, a vector's direction is inferred by the sign (e.g., 
``-" means the negative $x$ direction)
	  \item $\left| \vec{x} \right|$ = the magnitude or length of displacement vector
	  \item in 1D, $d \geq \left|\vec{x}\right|$ 
	  \item in 1D, $d = \left|\vec{x}\right|$ if there are no reversals in direction.  
	\end{enumerate}
\end{enumerate}
\item Average speed vs average velocity
  \begin{enumerate}
  \item Average speed = $\bar{v} = \frac{path~length}{\Delta t}=
\frac{d}{\Delta t}$
  \item Average velocity = $\bar{\vec{v}_x} = \frac{displacement}{\Delta t}=
\frac{\Delta \vec{x}}{\Delta t}$
  \item [Ex.] Mo walks 5\'\ left, 6\'\ right, 1\'\ left, and 8\'\ right in 30 seconds.  
What were his average speed and velocity? Ans. 0.66\'\/s and +0.27\'\/s $\hat{x}$.
\end{enumerate}
\item Instantaneous velocity and speed
  \begin{enumerate}
  \item instantaneous velocity: $\vec{v}_{inst}=^{\rm ~~~lim}_{\Delta t \rightarrow 0} = \frac{\Delta \vec{x}}{\Delta t}$
  \item instantaneous speed: $v_{inst} = \left| \vec{v}_{inst} \right|$
  \item [Note:] This goes against previous trend ...
  \item [Q:] Is $d=\left| \Delta \vec{x} \right|$? (Is distance just the magnitude of displacement?) Ans: No
  \item [Q:] Is $v_{avg} = \left| \vec{v}_{avg} \right|$ ?  Ans: No
  \item [Q:] Is $v_{inst} = \left| \vec{v}_{inst} \right|$ ?  Ans: yes
  \item Example: P.6.
  \item Example: P.9.
  \end{enumerate}
\item Acceleration = rate of change of velocity
  \begin{enumerate}
  \item Average acceleration:  $\bar{\vec{a}} = \frac{\Delta \vec{v}}{\Delta t}$
  \item Instantaneous acceleration: $\vec{a} = ^{\rm ~~~lim}_{\Delta t \rightarrow 0} = \frac{\Delta \vec{v}}{\Delta t} $
  \item Acceleration is caused by a force. $\vec{a} \propto \vec{F}$.
  \item Direction of $\vec{a}$ is indicated by sign.
    \begin{itemize}
    \item $\vec{a} > 0 \longrightarrow \vec{a}$ points in $+x$ direction.
    \item $\vec{a} < 0 \longrightarrow \vec{a}$ points in $-x$ direction.
    \item If $\vec{a}$ and $\vec{v}$ are in same direction, object is speeding up
    \item If $\vec{a}$ and $\vec{v}$ are in opposite direction, object is slowing up
    \item [Ex.] A car positioned left of the origin is speeding up as it 
	travels to the right.  What are the signs of $x$, $v$, and $a$? Ans:
        -, +, and +.
    \item [Ex.] P.17.
    \end{itemize}
  \end{enumerate}
\item Motion diagrams: pictures showing an objects x, v and a at evenly spaced
intervals of time.
\item One-dimensional motion with constant acceleration - equations
  \begin{enumerate}
  \item Velocity equations
	\begin{enumerate}
	\item $\vec{v}=\frac{d\vec{x}}{d t}$ ~~~~~(for any acceleration)
        \item $\vec{v}=\vec{a}t + \vec{v_i}$ ~~~~~(uniform a)
        \item $\bar{\vec{v}}=\frac{v_f + v_i}{2}$ ~~~~~(uniform a)
	\end{enumerate}
  \item Position equations
	\begin{enumerate}
	\item $\Delta x=\int_{t_i}^{t_f}v dt$  ~~~~~(any acceleration)
        \item $x(t) = \frac{1}{2}at^2 + v_i t + x_i$  ~~~~~(uniform a)
	\item $\Delta x = x_f - x_i = \bar{v}t$, so $x_f=x_i+(\frac{v_i+v_f}{2})t$ ~~~~~(uniform a)
	\item $v_f^2 - v_i^2 = 2 a_x (x_f - x_i)$ ~~~~~(uniform a)
	\end{enumerate}
  \item Use these equations for 1D motion problems with uniform acceleration.
	\begin{itemize}
	\item [Ex.] P. 23.
	\item Free fall problems. ($\left|a\right| = g = 9.8 m/s^2$)
	\item [Ex.] P. 43.
	\end{itemize}
  \end{enumerate}
\end{enumerate}

\end{document}
