\documentclass[12pt,thmsa]{article}
%%%%%%%%%%%%%%%%%%%%%%%%%%%%%%%%%%%%%%%%%%%%%%%%%%%%%%%%%%%%%%%%%%%%%%%%%%%%%%%%%%%%%%%%%%%%%%%%%%%%%%%%%%%%%%%%%%%%%%%%%%%%

\input astro.tex

\setlength{\topmargin}{-0.5in}
\setlength{\oddsidemargin}{-0.4in}
\setlength{\textwidth}{7in}
\setlength{\textheight}{9in}


\begin{document}


\begin{center}
{\huge The Moon and Eclipses }

{\Large Dr. J. Pinkney}

% {\Large Updated 2/20/2013 }\bigskip
\end{center}

\begin{enumerate}

\item The Moon moves its own diameter relative
to the stars in about \underline{~~~~~~~~~~~~~~~~~~~~~~~~~~~~~~~~~}
\begin{enumerate}
\item .55 minutes
\item 1900 arcseconds
\item 1 hour
\item .00015 hours
\item 13.2 days
\end{enumerate}

\item  The Earth is about how many times larger than the Moon (in diameter)?

\begin{enumerate}
\item  4\qquad (b) 10\qquad (c) 45 \qquad (d) 110
\end{enumerate}

\item  If you are looking at the Moon above the Southern horizon and the
right hand side of the Moon is less than half lit, the phase is

\begin{enumerate}
\item  waxing crescent\qquad (b) waxing gibbous \qquad (c) waning
gibbous\qquad (d) waning crescent
\end{enumerate}

\item  Which 26,000 year cycle is caused by the pull of the Moon and Sun 
on the Earth's equatorial bulge?
\begin{enumerate}
\item tug of war\qquad (b) lunar tides \qquad (c) jubilation \qquad (d)
eclipses\qquad (e) precession
\end{enumerate}

\item  Which phenomenon is caused by the difference in the gravitational
pull of the Moon on the near and far sides of the Earth?
\begin{enumerate}
\item  tug of war\qquad (b) lunar tides \qquad (c) precession \qquad (d)
eclipses\qquad (e) jubilation
\end{enumerate}

\item  Since it takes 29.5 days for the Moon to complete its phases, the
minimum time between two (penumbral) lunar eclipses is...

\begin{enumerate}
\item  about 2 weeks \qquad (b) about 1 year\qquad (c) about 6 months \qquad
(d) 60 days \\ (e) about 1 month
\end{enumerate}

\item  The Saros cycle is

\begin{enumerate}
\item  the time between extinctions\qquad (b) about 1 year \qquad (c) 
% NO: the time it takes for the Moon's line-of-nodes to rotate 360 degrees 
about 18 yrs 11.33 days long \qquad (d)
the time between solar eclipses \qquad (e) the synodic period of the planet
Saros.
\end{enumerate}

\item  About what time does the moon rise when its phase is new moon?

\begin{enumerate}
\item  6 am\qquad (b)12 pm \qquad (c) 6 pm\qquad (d) 12 am (midnight)\qquad
(e) 9 am
\end{enumerate}

\item  Lunar eclipses only occur during which phase of the Moon?

\begin{enumerate}
\item  \ New Moon\qquad (b) 1st quarter\qquad (c) Full Moon\qquad (d) 3rd
quarter\qquad
\end{enumerate}

\newpage

\item  Solar eclipses only occur during which phase of the Moon?

\begin{enumerate}
\item  \ New Moon\qquad (b) 1st quarter\qquad (c) Full Moon\qquad (d) 3rd
quarter\qquad
\end{enumerate}

\item  The distance between the Moon and the Earth is how many times larger
than the size of the Moon?

\begin{enumerate}
\item  \ 4\qquad (b) 10\qquad (c) 45\qquad (d) 110\qquad
\end{enumerate}

\item  Which is longer, the time it takes for the Earth to rotate
relative to the stars (sidereal day), or the time to rotate relative
to the Sun (solar day)?
% the sidereal month (time it takes Moon to line up
% with the stars) or the synodic month (time to line up with the Sun)?
\begin{enumerate}
\item  \ sidereal\qquad (b) solar \qquad (c) celestial\qquad (d) a and b
are the same \\
(e) the Earth doesn't rotate or else we would feel a wind.
\end{enumerate}

\item  Which is longer, the sidereal month (time it takes Moon to line up
with the stars) or the synodic month (time to line up with the Sun)?

\begin{enumerate}
\item  \ sidereal\qquad (b) synodic\qquad (c) celestial\qquad (d) a and b
are the same\qquad
\end{enumerate}

\item  Eclipse seasons, the 38 day period when eclipses can occur, are about
how many months apart?

\begin{enumerate}
\item  \ 2\qquad (b) 3\qquad (c) 6\qquad (d) 12\qquad
\end{enumerate}

\item  It is no coincidence that word ``ecliptic'' sounds like ``eclipse''.
This is because

\begin{enumerate}
\item  both look like big ``lips''
\item  there must be an eclipse when the Moon crosses the ecliptic.
\item  eclipses only occur when the Moon is near the ecliptic
\item  the shadow of the Moon follows the ecliptic
\item  the Moon's orbit is elliptical
\end{enumerate}

\item  The darkest portion of a shadow formed by a planet or moon is called
the \rule{0.5in}{0.02in}?

\begin{enumerate}
\item  \ cone\qquad (b) umbra\qquad (c) apex\qquad (d) penumbra\qquad
\end{enumerate}

\item  During a total lunar eclipse, the Moon can still be seen because of
reddish light from the \rule{0.5in}{0.02in}?

\begin{enumerate}
\item  \ Earth's atmosphere\qquad (b) Earth's street lights\qquad (c)
stars\qquad (d) lava on the Moon\qquad
\end{enumerate}

\item  Observing eclipses is an effective way to discover

\begin{enumerate}
\item  the shape of the Earth\qquad (b) the position of the ecliptic\qquad
(c) the Moon's orbit is not a perfect circle\qquad (d) the Sun's corona
\qquad (e) all of the above
\end{enumerate}

\item Suppose an annular solar eclipse is expected today.  At what time would
you expect the Moon to rise?
 \begin{enumerate}
\item at 12 pm (noon) \qquad (b) at sunrise \qquad (c) at sunset
\quad (d) at 12 am (midnight) \qquad (e) at 6 pm
\end{enumerate}

\item Suppose an total lunar eclipse is expected today.  At what time would
you expect the Moon to rise?
 \begin{enumerate}
\item at 12 pm (noon) \qquad (b) at sunrise \qquad (c) at sunset
\quad (d) at 12 am (midnight) \qquad (e) at 6 pm
\end{enumerate}

\item  After one \rule{0.9in}{0.02in} of 18 years 11.33 days, the same type
of eclipse repeats at about the same distance and position relative to the stars
and lunar nodes.  

\begin{enumerate}
\item  \ sidereal period\qquad (b) saros cycle\qquad (c) precession\qquad
(d) synodic period\qquad
\end{enumerate}

\item  Since we sometimes see annular solar eclipses instead of total solar
eclipses, we know \rule{0.6in}{0.01in}

\begin{enumerate}
\item  the position of the ecliptic\qquad (b) that the Moon is a
sphere\qquad (c) that the Moon rotates\qquad (d) that the Earth-Moon
distance is not constant\qquad (e) nothing
\end{enumerate}

\item  How does the plane of the Moon's orbit relate to the plane of the
Earth's orbit around the Sun?

\begin{enumerate}
\item  coincident (the same)\qquad (b) parallel\qquad (c) intersect with a 5$%
^{\circ }$ angle\qquad (d) intersect with a 23.5$^{\circ }$ angle\qquad (e)
perpendicular
\end{enumerate}

\item (1pt) How does the parallax angle $p$ of a star depend on the distance $D$
to the star?
\begin{enumerate}
\item the bigger $D$ the bigger $p$ \qquad (b) the bigger $D$ the smaller $p$
\qquad (c) no dependence
\end{enumerate}

\item The formula $d=\frac{1}{p}$ gives the distance
measured in \underline{~~~~~~~~~~~~~~~~~~~~~~~~~~~~~}
to an object with a parallax angle measured in arcseconds.
\end{enumerate}

\section{Historical Astronomy }

\begin{enumerate}

\item An astronomical observatory/temple built by the Mayan's is called
\begin{enumerate}
        \item the Big Horn Medicine Wheel
        \item Caracol
        \item Stonehenge
        \item the Colloseum
        \item Quetzalquatl
\end{enumerate}

\item T or F. It was the Chinese who provided critical ancient records of comets.
% and supernovae

\item T or F. Like the Sun and the Moon, the planets usually move from west to east (rel to the stars) from one day to the next.

\item  The ancient people credited with creating the astrology used today is

\begin{enumerate}
\item  the Babylonians\qquad (b) the Chinese\qquad (c) the Plains
indians\qquad (d) the Polynesians \qquad (e) the Norwegians
\end{enumerate}

\item  The ``calendar'' made out of rock slabs which is located on the
British Isles is called

\begin{enumerate}
\item  Big Horn Medicine Wheel\qquad (b) Caracol\qquad (c) Stonehenge\qquad
(d) Buckminster Abbey \qquad (e) Montezuma's revenge
\end{enumerate}

\item  The work of the ancient Greeks was not forgotten during the dark ages
largely because of the

\begin{enumerate}
\item  Babylonians\qquad (b) Islamic peoples\qquad (c) Native
Americans\qquad (d) Egyptians \\ (e) Mayans
\end{enumerate}

\item The ``luminaries" to the Greeks and Romans included the five
known planets and the \underline{~~~~~~~~~~~~~~~~~} and
\underline{~~~~~~~~~~~~~~~~~~~~}.

\end{enumerate}

\end{document}
